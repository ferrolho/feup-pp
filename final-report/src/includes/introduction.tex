\section{Introdução}

No âmbito da Unidade Curricular de Paradigmas da Programação do Programa Doutoral em Engenharia Informática foi proposto o desenvolvimento de uma plataforma de identificação de grupos de co-expressão entre todos os genes de proteínas mitocondriais.

\bigskip

A plataforma pode ser vista como um motor de busca a uma base de dados com diversos tecidos humanos, que contém pares de genes relevantes de cada tecido e a sua respetiva semelhança. Existem filtros para refinar os resultados de pesquisa: um para selecionar um dado intervalo de semelhança entre genes, e outro para a ordenação do coeficiente de semelhança.

A ferramenta deve por isso ser usada como um auxílio à investigação, poupando tempo aos investigadores, que podem tirar conclusões com base na informação presente na base de dados muito mais rapidamente do que se tivessem que gerir toda essa informação manualmente num sistema não especializado.

\bigskip

O interesse numa ferramenta desta escala não se limita meramente ao nível académico, mas também a centros de investigação, como por exemplo o Instituto de Patologia e Imunologia Molecular da Universidade do Porto - IPATIMUP.

\newpage
